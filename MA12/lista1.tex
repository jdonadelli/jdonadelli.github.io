%Time-stamp: <seg mar 26 15:32:02 2018 yair@phasetransition>
\pdfoutput=1
%%%%%%%%%%%%%%%%%%%%%%%%%%%%%%%%%%%%%%%%%%%%%%%%%%%%%%%
\documentclass[10pt]{book} 
\usepackage[brazil]{babel}
\usepackage[T1]{fontenc}
\usepackage[utf8]{inputenc} 

\usepackage{multicol}
\usepackage{amssymb,amsmath,amsthm}
\usepackage{concrete}

\linespread{1.3}

%%%%%%%%%%% MACROS %%%%%%%%%%%%%%%%%%%%%%%%%%%%%%%%%%%%%%%%
\input{/home/yair/Dropbox/C_macros.tex}

\begin{document}             

\begin{center}
  \textbf{Lista 1 de Matemática Discreta (MA12) \\ Data de entrega: 05/04}
\end{center}

\bigskip


Resolva os exercícios abaixo \underline{usando o princípio de indução finita}.

\begin{enumerate}


\item Prove que a soma dos $n$ primeiros termos da p.a.~$\paren*{a_n}$
  de razão $r$ é $S_n = \frac{(a_1+a_n)n}2$.


\item Prove que a soma dos $n$ primeiros termos da p.g.~$\paren*{a_n}$
  de razão $q\neq 1$~é $S_n = a_1\frac{\;\; 1-q^n}{1-q}$.

  
\item Seja $\paren*{a_n}$ uma p.g.~com termo inicial $a_1>0$ e razão
  $r>1$. Prove que soma dos $n$ primeiros termos satisfaz
  \[ S_n \< \frac r{r-1}a_n\] para todo $n>1$.

\end{enumerate}


\end{document}



%%% Local Variables:
%%% mode: latex
%%% TeX-master: t
%%% End:
